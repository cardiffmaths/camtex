% cambi_test.tex
\PassOptionsToPackage{en}{cambi}
\PassOptionsToPackage{cy}{cambi}

\documentclass{article}

\usepackage{cambi}

\usepackage[a4paper,margin=20mm]{geometry}
\setlength{\parindent}{0ex}
\setlength{\parskip}{1ex}

\title{\mbox{}\en Multilingual Tools for \LaTeX\ \cy Offer Amlieithyddol ar gyfer \LaTeX\ }
\author{Dafydd Evans}

%===============================================================
\begin{document}
\maketitle

\begin{abstract}\noindent
The {\tt culang} package provided macros and environments for maintaining a single bilngual source file for which each version can be typeset separately. The main contribution is to define the marks \verb+\en+, \verb+\cy+ and \verb+\bi+ to switch between English, Welsh and bilngual language modes. Unmarked text will appear in all versions of the document.
\end{abstract}

%--------------------
\subsection*{\en Preface \cy Rhagair}

{\tt cambi.sty}: 
\cy Offer amlieithyddol ar gyfer \LaTeX\ 
\en Multilingual tools for \LaTeX

%----------------------------------------
\section{\en Macros \cy Macroau}

\en There are three macros\cy Mae tri macro\bi: 
\begin{itemize}
\item \verb+\eng{...}+, 
\item \verb+\wel{...}+ \bil{and}{a}
\item \verb+\bil{...}{...}+.
\end{itemize}

\en Test: \cy Prawf:\bi
\begin{enumerate}
\item \eng{This is the English version.}\wel{Dyma'r fersiwn Gymraeg.}
\item \bil{This is the English version.}{Dyma'r fersiwn Gymraeg.}
\end{enumerate}

%----------------------------------------
\section{\bil{Environments}{Amgylcheddau}}

\begin{english}
Contents of a \verb+\begin{english}...\end{english}+ environment.
\end{english}

\begin{welsh}
Cynnyws amgylchedd \verb+\begin{welsh}...\end{welsh}+.
\end{welsh}

\begin{cymraeg}
Cynnyws amgylchedd \verb+\begin{cymraeg}...\end{cymraeg}+.
\end{cymraeg}

%----------------------------------------
\section{\eng{Theorems}\wel{Theoremau}}
The welsh {\tt babel} language definitions cover basic macros (e.g. \verb+\abstractname+, \verb+\bibname+). The \verb+cambi+ package adds language definitions for theorem names and miscellaneous others.

\begin{theorem}
{\en There are infinitely many prime numbers.\cy Mae nifer anfeidrol o rifau cysefin.}
\end{theorem}	

\begin{exercise}
\en Simplify the expression \cy Symleiddiwch y mynegiant $x^2-2x+1$.
\end{exercise}	


%----------------------------------------
\section{\en Marks \cy Marciau}

The syntax \verb+{\bf bold text}+ is often more convenient than \verb+\textbf{bold text}+. 

We define \verb+\cy+ and \verb+\en+ to switch between the two languages.

\begin{itemize}
\item \verb+\cy+ switches the language mode to Cymraeg (Welsh).
\item \verb+\en+ switches the language mode to English.
\item Unmarked text will appear in all versions of the document. 
\item An explicit closing mark (\verb+\bi+) switches back to the `common' state.
\item The effect of \verb+\cy+ and \verb+\en+ is also terminated by \verb+\ {, }, $+ and newline. 
\end{itemize}

%----------------------------------------
\section{\eng{Marks test}\cym{Prawf marciau}}

\en Please see the source file for more details.
\cy Gweler y ffeil ffynhonnell am fwy o fanylion.

% Welsh version
\welshon % overrides package option
\subsection*{\cy Y fersiwn Gymraeg \en The English version}

\bigskip\hrule\bigskip

start \cy Helo Byd! \en Hello World! \cy Hwyl fawr! \en Goodbye! $\alpha$ end.

\bigskip\hrule\bigskip

start \en Hello World! \cy Helo Byd! \en Goodbye! \cy Hwyl fawr! $\alpha$ end.

\bigskip\hrule\bigskip

\cy Helo Byd! \en Hello World! \cy Hwyl fawr! \en Goodbye!
Next char.

Next line.

% English version
\welshoff % overrides package option
\subsection*{\cy Y fersiwn Gymraeg \en The English version}

\bigskip\hrule\bigskip

start \cy Helo Byd! \en Hello World! \cy Hwyl fawr! \en Goodbye! $\alpha$ end.

\bigskip\hrule\bigskip

start \en Hello World! \cy Helo Byd! \en Goodbye! \cy Hwyl fawr! $\alpha$ end.

\bigskip\hrule\bigskip

% hack: this needs a \strut at the beginning again!
\strut\cy Helo Byd! \en Hello World! \cy Hwyl fawr! \en Goodbye!
Next char.

Next line.

% reset to the chosen language
\resetlanguage

% the end
\bigskip
\hrule
\bigskip
\begin{center}\Large\scshape
\strut\en The End\cy Y Diwedd \bi
\normalsize\normalfont\end{center}
\bigskip
\hrule

\end{document}









