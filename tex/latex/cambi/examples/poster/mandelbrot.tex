% mandelbrot.tex
\PassOptionsToPackage{en}{cambi}
%\PassOptionsToPackage{cy}{cambi}

\documentclass{article}

\usepackage{cambi}

\usepackage[margin=0cm, papersize={10cm, 9cm}]{geometry}
\setlength{\parindent}{0ex}
\pagestyle{empty}

\usepackage{graphicx}
\graphicspath{{./figures/}}
\usepackage[font=small]{caption}

%------------------------------------------------
\begin{document}
\section*{\en The Mandelbrot Set \cy Set Mandelbrot}

\en Consider the following iterated system,
\cy Ystyriwch y system ailadroddol ganlynol,

\[
z_{n+1} = z_n^2 + c \quad\text{\en with\cy gyda}\quad z_0 = 0.
\]

\eng{The Mandelbrot set is the set of complex numbers $c$ for which $z_n$ does not diverge to infinity as $n\to\infty$.}
\cym{Set Mandelbrot yw'r set o rifau cymhlyg $c$ sydd fel nad ydyw $z_n$ yn dargyfeirio at anfeidredd pan mae $n\to\infty$.}

\begin{figure}[h]
\centering
\includegraphics[scale=0.2]{mandelbrotset}
\footnotesize\caption{\eng{The Mandelbrot set}\cym{Set Mandelbrot}}
\end{figure}

\en The elaborate boundary of the set is infinitely complex and exhibits self-similarity, and is therefore a fractal object.
\cy Mae ffin goeth y set yn anfeidrol gymhleth ac yn dangos hunan-guflunedd, ac mae felly yn wrthrych ffractal.

\end{document}


